% withpage: ページ番号をつける (著者確認用)
% english: 英語原稿用フォーマット
\documentclass{ipsjprosym}
%\documentclass[withpage,english]{ipsjprosym}

\usepackage[dvipdfmx]{graphicx}
\usepackage{latexsym}

\usepackage{listings}

\begin{document}

% Title, Author %%%%%%%%%%%%%%%%%%%%%%%%%%%%%%%%%
\title{環境にメソッドを直接格納する \\ 新しいオブジェクトシステムの提案}

\affiliate{COINS}{筑波大学情報学群情報科学類}
\affiliate{CS}{筑波大学システム情報系}

\author{林 拓人}{Takuto Hayashi}{COINS}[hayashi@ialab.cs.tsukuba.ac.jp]
\author{前田 敦司}{Atusi Maeda}{CS}[maeda@cs.tsukuba.ac.jp]

\begin{abstract}
%[概要(400字程度)]
従来のクラスベース・オブジェクトシステムは,メソッドをクラスに格納するものと総称関数に格納するもの
(メソッドがクラスに属すものと総称関数に属すもの)とに分けられる.
本論文はこれらのいずれとも異なり,環境にメソッドを直接格納する新しいオブジェクトシステムを
提案する.
クラスや総称関数と言った枠を廃し,クラス名とメソッド名の組をキーとして環境にメソッドを直接格納する.
これにより変数に対して行えるあらゆる操作がメソッドに対しても行えるようになり,従来の方式に比べ
シンプルな仕組みでより柔軟なオブジェクト指向プログラミングが可能となる.
提案する手法の有用性を実証するため,このオブジェクトシステムを搭載する独自言語Suzuの
処理系を実装した.
これが実際に言語内DSL(Domain Specific Language)の作成に有用であることを示す.
従来のオブジェクトシステムにおける類似した概念等との関連についても議論する.
\end{abstract}

\begin{jkeyword}
オブジェクトシステム,メソッド,クラス,総称関数,環境
\end{jkeyword}

\maketitle

% Body %%%%%%%%%%%%%%%%%%%%%%%%%%%%%%%%%
\section{序論}

従来のクラスベース・オブジェクトシステムは,メソッドの格納方式をモデル化すると
次の2つのモデルに分類することができる.
1つはクラスにメソッドを格納するモデル,もう1つはCLOS\cite{Ida:2010}のように
総称関数にメソッドを格納するモデルである.

本論文は,これら2つの格納モデルに対する分析に基づき考案した,全く新しい格納モデルを持つ
オブジェクトシステムを提案する.
また,その特徴を最大限に生かせるよう独自に設計したプログラミング言語Suzuを用いて,提案する
オブジェクトシステムの評価を行う.

\section{従来のオブジェクトシステム}

ここではモデルを単純化するため,単一ディスパッチの場合についてのみ考える.
多重ディスパッチへのモデルの拡張については\ref{sec:multiple-dispatch}節で検討する.

\subsection{クラスにメソッドを格納するモデル}
\label{sec:classes}

クラスにメソッドを格納するモデルでは環境にクラスを格納し,クラスにメソッドを格納する
(図\ref{fig:classes}).
ここで環境はクラス名をキーとしてクラスを格納する辞書であり,クラスはメソッド名をキーとして
メソッドを格納する辞書である.
SmalltalkやRubyといった言語のオブジェクトシステムはこのモデルに分類される.

\subsection{総称関数にメソッドを格納するモデル}
\label{sec:generic-finctions}

総称関数にメソッドを格納するモデルでは環境に総称関数を格納し,総称関数にメソッドを格納する
(図\ref{fig:generic-functions}).
ここで環境はメソッド名をキーとして総称関数を格納する辞書であり,総称関数はクラス名をキーとして
メソッドを格納する辞書である.
単一ディスパッチに限定したCLOSはこのモデルに分類される.

\section{提案するオブジェクトシステム}
\label{sec:proposal}

従来のオブジェクトシステムの分類先である2つのモデルは,どちらもクラス名とメソッド名が決まれば
メソッドが一意に定まるという特徴を持つ.
また,環境という辞書の中にクラスあるいは総称関数という辞書が入れ子になっている構造も
共通している.

提案するオブジェクトシステムはこのようなクラスや総称関数による入れ子構造を廃し,
\textbf{環境にメソッドを直接格納するモデル}を採用する(図\ref{fig:environment}).
ここで環境は\textbf{クラス名とメソッド名の組}をキーとしてメソッドを格納する辞書である.

このモデルの特徴は,メソッドの格納方式が一般的な変数の格納方式と類似していることである.
メソッドがクラス名とメソッド名の組をキーとして環境に格納されるのに対し,変数は変数名を
キーとして環境に格納される.

これはすなわち,「変数名」を「クラス名とメソッド名の組」に置き換えることで,
\textbf{変数に対して行えるあらゆる操作がメソッドに対しても行える}ということである.
具体的には,ローカル変数に対応する\textbf{ローカルメソッド}の定義,シャドーイング,
モジュールからのエクスポート・インポート,仮引数やパターンマッチングのパターンとしての指定などが
挙げられる.
\ref{sec:implementation}節では提案するオブジェクトシステムを搭載した独自の
プログラミング言語Suzuを用いて,この特徴を生かした実際のプログラム例を示す.

\begin{figure}
\centering
\includegraphics[width=7cm]{fig/classes-crop.pdf}
\caption{クラスにメソッドを格納するモデル}
\label{fig:classes}
\end{figure}

\begin{figure}
\centering
\includegraphics[width=7cm]{fig/generic-functions-crop.pdf}
\caption{総称関数にメソッドを格納するモデル}
\label{fig:generic-functions}
\end{figure}

\begin{figure}
\centering
\includegraphics[width=7cm]{fig/environment-crop.pdf}
\caption{環境にメソッドを直接格納するモデル}
\label{fig:environment}
\end{figure}

\section{実装:プログラミング言語Suzu}
\label{sec:implementation}

環境にメソッドを直接格納するモデルに基づき設計した独自のプログラミング言語Suzuの解説とその
プログラム例により,提案するオブジェクトシステムの有用性を示す.

\subsection{基本的な文法}

\verb|//|から行末まではコメントである.
以降,\verb|//=>|に続くコメントはプログラムの出力を表すものとする.
出力にはデバッグ用出力関数\verb|p|を用いる.

変数を定義するには\verb|let x = 123|のようにする.
ここでは変数\verb|x|に整数値\verb|123|を代入している.

関数呼び出しは\verb|func(arg1, arg2)|のようにする.
ここでは関数\verb|func|を第1引数\verb|arg1|,第2引数\verb|arg2|で呼び出している.

関数定義は以下のようにする.
\begin{quote}
\begin{verbatim}
def fac(n):
  if(n == 0):
    1
  else:
    n * fac(n - 1)
  end
end
\end{verbatim}
\end{quote}
これは以下のコードと意味的に等価である.
\begin{quote}
\begin{verbatim}
let fac = ^(n):
  if(n == 0):
    1
  else:
    n * fac(n - 1)
  end
end
\end{verbatim}
\end{quote}
\verb|^(n):|から\verb|end|までの部分は引数\verb|n|を受け取って\verb|end|までの
式を評価する関数リテラルである.\verb|^(n){ ... }|と書くこともできる.

関数リテラルは関数呼び出しの後ろに続けて書くことで追加の引数として高階関数に渡すことができる.
例えば
\begin{quote}
\begin{verbatim}
map(lst)^(n):
  n * 2
end
\end{verbatim}
\end{quote}
は
\begin{quote}
\begin{verbatim}
map(lst, ^(n){ n * 2 })
\end{verbatim}
\end{quote}
と等価である.

\verb|begin|は新たなスコープを導入し\verb|end|までの式を評価する.
\begin{quote}
\begin{verbatim}
let str = "global"
begin:
  let str = "local"
  p(str)  //=> "local"
end
p(str)  //=> "global"
\end{verbatim}
\end{quote}
一般に,\verb|:|と\verb|end|(または\verb|{|と\verb|}|)で囲まれた箇所には
新たなスコープが導入される.

\subsection{オブジェクトシステムの解説}

Suzuは\ref{sec:proposal}節で提示したモデルを素朴に実装している.
例えばクラス名とメソッド名の組は\verb|C#m|と書く.\verb|C|はクラス名,\verb|m|は
メソッド名である.

クラス定義は以下のようにして行う.
\begin{quote}
\begin{verbatim}
class Vector:
  def MkVector(x, y)
end
\end{verbatim}
\end{quote}
これにより,クラス\verb|Vector|とそのコンストラクタ\verb|MkVector|が定義される.

メソッドはクラスは別に以下のように定義する.
\begin{quote}
\begin{verbatim}
def Vector#add(MkVector(x1, y1),
               MkVector(x2, y2):
  MkVector(x1 + x2, y1 + y2)
end
\end{verbatim}
\end{quote}
通常の関数定義における変数名をクラス名とメソッド名の組に置き換えた形となっている.
変数と同様\verb|let|を用いた書き方も可能である.

メソッドは\verb|obj.m(arg1, arg2)|のようにして呼び出す.
メソッドが呼び出される際は第1引数として呼び出しの対象となったオブジェクト(レシーバ)が渡される.
\begin{quote}
\begin{verbatim}
let a = MkVector(1, 2)
let b = MkVector(3, 4)
p(a.add(b))  //=> MkVector(4, 6)
\end{verbatim}
\end{quote}
また,独自の演算子をメソッドとして定義することもできる.
\begin{quote}
\begin{verbatim}
let Vector#(+) = Vector#add
p(a + b)  //=> MkVector(4, 6)
\end{verbatim}
\end{quote}

Suzuのオブジェクトシステムは環境にメソッドを格納するため,ローカル環境にメソッドを定義する
ことが可能だ.
ここで言うローカル環境とは,ブロック(制御構文でインデントを行う部分),関数,
モジュールといったあらゆる局所的なスコープのことである.
\begin{quote}
\begin{verbatim}
def Vector#m(self):
  write_line("global")
end
begin:
  def Vector#m(self):
    write_line("local")
  end
  d.m  //=> local
end
d.m  //=> global
\end{verbatim}
\end{quote}

\subsection{その他の特徴}

Suzuのモジュールは,他の言語における変数のようにメソッドを個別にエクスポートできる.
\begin{quote}
\begin{verbatim}
module A:
  class C = ...
    ...
  end
  def f(...):
    ...
  end
  def C#m(...):
    ...
  end
  def C#n(...):
    ...
  end
  export f, C#m
end
\end{verbatim}
\end{quote}
上の例ではモジュール\verb|A|は関数\verb|f|とメソッド\verb|C#m|をエクスポートするが,
\verb|C#n|はエクスポートしない.エクスポートされた変数を参照するには\verb|A::f|のようにする.

モジュールは局所的に\verb|open|することができる.
モジュールを\verb|open|するとエクスポートされている変数やメソッドがそのスコープで
直接定義されたようにインポートすることができる.
また,\verb|except|キーワードを用いてインポートの対象から除外することもできる.
\begin{quote}
\begin{verbatim}
module B:
  ...
  export f, g, C, C#m, C#n
end
begin:
  open B except g, C#n
  ...
end
\end{verbatim}
\end{quote}
上の例では\verb|begin|から\verb|end|までのスコープへ,
モジュール\verb|B|でエクスポートされているもののうち\verb|g|と\verb|C#n|を除いて
インポートを行う.

Suzuは継承機構を持たず,実装の再利用はトレイト\cite{Scharli:2003}を用いて行う.
ただし,Suzuのトレイトはそのオブジェクトシステムに適合するよう他の言語と比べて多少
趣の異なるものとなっている.
\begin{quote}
\begin{verbatim}
trait T(C, C#m, C#n):
  def C#o(self, ...):
    ...
    self.m(...)
    ...
  end
  def C#p(self, ...):
    ...
    self.n(...)
    ...
  end
  export C#o, C#p
end
\end{verbatim}
\end{quote}
トレイトは\textbf{クラスやメソッドを受け取ってモジュールを返す関数}として表現される.
使用する側は以下のようにする.
\begin{quote}
\begin{verbatim}
open T(D, D#m, D#n)
open T(E, E#m, E#n)
\end{verbatim}
\end{quote}
これにより,\verb|D#o|,\verb|D#p|,\verb|E#o|,\verb|E#p|が定義される.
トレイト同士の加算は単に\verb|open|を並べればよい.メソッドの減算は\verb|except|を用いる.
メソッドのリネームは\verb|except|に\verb|let|を用いた個別のインポートを組み合わせる.
例えばトレイト \verb|S|によって定義されるメソッド\verb|m|を\verb|n|にリネームしたい場合,
以下のようにする.
\begin{quote}
\begin{verbatim}
open S(C) except C#m
let C#n = S(C)::(C#m)
\end{verbatim}
\end{quote}

\subsection{プログラム例:PEGパーザコンビネータ}

Suzuの有用性を示すプログラム例として,PEGパーザコンビネータライブラリを作成した.
ソースコードは付録の\ref{sec:peg-source}節に記載している.

このライブラリを用いると,パーザを言語内DSLとして簡潔に記述することができる.
例えば,文脈自由でない言語$\{a^n b^n c^n:n\ge1\}$を表すPEG
\begin{quote}
\begin{verbatim}
S <- &(A !'b') 'a'+ B !'c'
A <- 'a' A? 'b'
B <- 'b' B? 'c'
\end{verbatim}
\end{quote}
のパーザは,\verb|PEG::New|を使用して以下のように書ける.
\begin{quote}
\begin{verbatim}
let s = begin:
  open PEG::New()
  "S" <- and(nt_ref("A") &+
             not(char('b')))
      &+ one_or_more(char('a'))
      &+ nt_ref("B")
      &+ not(char('c'))
  "A" <- char('a')
      &+ zero_or_one(nt_ref("A"))
      &+ char('b')
  "B" <- char('b')
      &+ zero_or_one(nt_ref("B"))
      &+ char('c')
  nt_ref("S")
end
\end{verbatim}
\end{quote}
\verb|PEG::New|の呼び出しの戻り値を\verb|open|することで,文法定義のための関数群が
現在のスコープにインポートされる.
例えば,\verb|(<-)|は非終端記号の定義,\verb|nt_ref|は非終端記号の参照,
\verb|(&+)|は連接,\verb/(|+)/は選択,\verb|and|は肯定先読み,\verb|not|は
否定先読みである.
パーザは\verb|PEG::parse|を使用して実行できる.
\begin{quote}
\begin{verbatim}
p(PEG::parse(s, ""))
  //=> Failure()
p(PEG::parse(s, "abc"))
  //=> Success(...)
p(PEG::parse(s, "ab"))
  //=> Failure()
p(PEG::parse(s, "aaabbbccc"))
  //=> Success(...)
p(PEG::parse(s, "aabbbccc"))
  //=> Failure()
\end{verbatim}
\end{quote}

このライブラリは,\textbf{既存のデータ型に対して新たな演算子をローカルに定義できる}という
Suzuの特徴を生かしている.
演算子\verb|(<-)|の正体はメソッド\verb|String#(<-)|であり,スコープを限定した上で
既存の文字列クラス\verb|String|に対し新たに定義されている.

このように,Suzuのオブジェクトシステムは特定のスコープでのみ有効なメソッドを定義
できることで,モジュール性の高い言語内DSLの作成を可能にしている.

\section{関連研究}

従来のオブジェクトシステムにSuzuと似た柔軟性を与える取り組みや,
構造上の類似性を持つ概念との比較を行う.

ContextJ\cite{AppeltauerMalte:2011}は,文脈指向プログラミングにおけるlayerという
概念に基づいたJavaの拡張言語である.
layerを切り替えるによりメソッドの定義を切り替えられる点がSuzuのモジュールと類似しているが,
ContextJのメソッド定義はJavaと同様クラスの内部でしか行えない.

GluonJ\cite{Chiba:2010:MMC:1869459.1869503}は,Javaでアスペクト指向
プログラミングを行うためのシステムである.
Glueと呼ばれるクラスを定義することで既存のクラスの外部でメソッドを再定義できるが,
再定義の影響範囲は後述のClassboxes\cite{Bergel:2005:CCV:1646591.1646599}と違い
グローバルである.

Classboxesは,影響範囲をClassboxesというモジュール内に制限してメソッドを再定義できる
システムである.
SuzuはClassboxesのようなクラス専用の特殊なモジュールを使うことなく,変数を扱うのと
同じモジュールシステムによって再定義の影響範囲を制限できる.
また,Classboxesはダイナミックスコープだが,Suzuのモジュールはレキシカルスコープである.
このためSuzuはClassboxesがサポートしているlocal rebindingという機能を
サポートしていない.

Method Shells\cite{Takeshita:2014-07-14}は,linkとincludeという2種類の宣言を
使い分けることによって,Classboxesで生じるlocal rebindingの問題を回避できる
モジュールシステムである.
先ほど述べたようにSuzuはlocal rebindingをサポートしない.local rebindingは
Method Shellsの論文で指摘されているような問題を生じさせること,これを回避するために
Method Shellsのような工夫が必要であることから,そのような工夫を必要とするよりは
そもそもlocal rebindingをサポートせず仕組みをシンプルにすることを選択した.

Refinements\cite{Maeda:2013}は,プログラミング言語Rubyに導入されたClassboxesと
類似する機構である.
Refinementsはレキシカルスコープであり,local rebindingをサポートしない点で
Suzuのモジュールシステムにより近い.
しかしながら,Refinementsはメソッドの再定義の有効・無効をファイル単位でしか制御できない.
Suzuはより細かいブロック単位での制御が可能である.

Suzuのオブジェクトシステムはデータ型の定義の外部でその値に対する演算子の振る舞いを
変えられるという点で,型クラス\cite{Wadler:1989:MAP:75277.75283}に類似している.
Suzuは値に紐付けられたクラスという動的な性質によって演算子の振る舞いを変えるが,
型クラスは式に紐付けられた型という静的な性質によって演算子の振る舞いを変える.
型クラスは導入に静的な型を必要とするが,Suzuはこれを必要としない.
また型クラスのインスタンス宣言はモジュールからのエクスポート・インポートによって可視性を
制御できないが,Suzuのメソッド定義はこれが可能である.
ただし型クラスは戻り値の型に応じて関数の振る舞いを変えさせることができるのに対し,
Suzuではこれは不可能である.

MixJuice\cite{Ichisugi:2002}は,複数のモジュールにクラス定義を分割し組み合わせることで,
プログラムのモジュール化を促進するプログラミング言語である.
MixJuiceのモジュールシステムが与える柔軟性はSuzuのそれと非常に類似している.
違いはMixJuiceが\ref{sec:classes}節で述べたクラスにメソッドを格納するモデルを採用している
のに対し,Suzuは\ref{sec:proposal}節の環境にメソッドを直接格納するモデルを採用している
ことである.
ローカル変数に対応するローカルメソッドを定義できることや,仮引数・パターンマッチングの
パターンとしてメソッドを指定できることは,唯一Suzuのみが持つ特徴である.

\section{今後の課題}

\subsection{継承機構}

Suzuは継承機構を持たず,トレイトによって実装の再利用を行う.
継承機構を持たないことによって生じる不都合については今後検討していく必要がある.

もしSuzuに継承機構を追加するならば,オブジェクトに複数のクラス名を持たせればよい.
クラス名は先頭からメソッド解決の順に並べたリストで持つようにする.
メソッド呼び出しの際はオブジェクトが持つリストの先頭から順にクラス名を取り出し,
メソッド名との組を作ってこれをキーとし環境からメソッドの探索を行う.

単一継承に制限する場合,クラス名のリストは先頭が継承ツリーの最下位クラス,
末尾が最上位クラスとなる.
多重継承を許す場合,C3線形化\cite{Barrett:1996:MSL:236337.236343}等を用いて
適切な順序で並べたクラス名のリストを生成する必要がある.

\subsection{多重ディスパッチ}
\label{sec:multiple-dispatch}

\ref{sec:generic-finctions}節で示した総称関数にメソッドを格納するモデルは,
CLOSがサポートする多重ディスパッチに対応していない.
モデルを多重ディスパッチに対応させるには,総称関数を単なる辞書ではなくある種のデータベース
としてとらえる必要がある.
データベースはすべての引数のクラス名を受け取って,その組み合わせにマッチするメソッドを検索し返す.

Suzuは多重ディスパッチに対応していないが,このモデルを応用し対応させることが可能である.
すなわち環境をある種のデータベースとしてとらえ,複数のクラス名と1つのメソッド名を受け取って
その組み合わせにマッチするメソッドを検索し返す.

つまり,1つのクラス名と1つのメソッド名からメソッドが決まるのが単一ディスパッチ,
複数のクラス名と1つのメソッド名からメソッドが決まるのが多重ディスパッチであると言える.
ここで自然と,1つのクラス名と複数のメソッド名または複数のクラス名と複数のメソッド名から
メソッドが決まるシステムというのも思い浮かぶ.
これらは既存のオブジェクトシステムにない概念であり,考察の余地がある.

\subsection{メソッド呼び出しの最適化}

Suzuのオブジェクトシステムにはメソッド呼び出しの一般的な最適化手法がそのままでは適用できない
ことがある.
これはSuzuがメソッドをローカルに定義できることや,関数の仮引数としてメソッドを指定できることによる.
同じクラス名とメソッド名の組に対しても呼び出し位置が変われば呼び出されるメソッドが変わるほか,
同じ位置においても関数呼び出しのたびにメソッドの内容が変わることもある.

比較的容易に適用できる最適化としてはインラインキャッシュ\cite{Onodera:1997-04-15}がある.
メソッドの呼び出し位置ごとに仮想マシン命令の内部にキャッシュを用意するため,
メソッドのローカル定義にも対応できる.
ただし,仮引数として指定されたメソッドのように関数呼び出しのたびにメソッドの内容が変わることもあるため,
メソッドの値を直接キャッシュしてはならない.
関数呼び出しでのメソッドの格納位置を固定し,格納位置を指すインデックスをキャッシュする必要がある.

現在Suzuの処理系は特に最適化を施していないため,適切な最適化手法を考え実装することが
課題である.

\section{結論}

従来のクラスベース・オブジェクトシステムはクラスにメソッドを格納するモデルと
総称関数にメソッドを格納するモデルとに分類されたが,そのどちらにも属さない,
環境にメソッドを直接格納するモデルを持つオブジェクトシステムを提案した.

このモデルにおいては変数に対して行えるあらゆる操作がメソッドに対しても行える.
この特徴を生かし,提案するオブジェクトシステムを搭載した独自言語Suzuを用いて
可読性の高い言語内DSLを作成し,提案手法の有用性を示した.

オブジェクト指向プログラミングにおける既存の様々な概念の整理にも役立った.
今後はより実用性を意識した拡張や効率的な実装について考えていくことが課題である.

%\begin{acknowledgment}
%謝辞が必要であれば,ここに書く.
%\end{acknowledgment}

% BibTeX を使用する場合 %%%%%%%%%%%%%%%%%%%%%%%%%%%%%%%%%
\bibliographystyle{ipsjsort}
\bibliography{prosym}

% BibTeX を使用しない場合
%\begin{thebibliography}{9}
%\bibitem{latex} 奥村晴彦, 黒木裕介: \textbf{LaTeX2e美文書作成入門}. 技術評論社, 
%2013.
%\end{thebibliography}

\onecolumn

\appendix

\section{PEGパーザコンビネータライブラリのソースコード}
\label{sec:peg-source}

\begin{lstlisting}[
  basicstyle=\scriptsize\ttfamily,
  breaklines=true,
  lineskip=-2.4ex,
  ]
module PEG:
  class Expr:
    def MkExpr(proc)
  end

  class Result:
    def Success(pos, value)
    def Failure()
  end

  def parse(MkExpr(expr), str):
    expr(str, 0, Hash::create(16))
  end

  trait New():
    let nonterms = Hash::create(16)
    
    def def_nt(name, MkExpr(expr)):
      nonterms[name] = expr
    end

    def char(c):
      MkExpr^(str, pos, caches):
        if(pos < String::length(str) && str[pos] == c):
          Success(pos + 1, c)
        else:
          Failure()
        end
      end
    end

    def string(s):
      let len = String::length(s)
      MkExpr^(str, pos, caches):
        if(pos + len <= String::length(str) && String::sub(str, pos, len) == s):
          Success(pos + len, s)
        else:
          Failure()
        end
      end
    end

    def char_set(cs):
      MkExpr^(str, pos, caches):
        if(pos < String::length(str) && String::contain?(cs, str[pos])):
          Success(pos + 1, str[pos])
        else:
          Failure()
        end
      end
    end

    def nt_ref(name):
      MkExpr^(str, pos, caches):
        match(Hash::get(caches, (name, pos))):
        case(Some(result)):
          result
        case(None()):
          let result = nonterms[name](str, pos, caches)
          caches[(name, pos)] = result
          result
        end
      end
    end

    let fail = MkExpr^(str, pos, caches):
      Failure()
    end

    let no_op = MkExpr^(str, pos, caches):
      Success(pos, ())
    end

    def seq(MkExpr(expr1), MkExpr(expr2)):
      MkExpr^(str, pos, caches):
        match(expr1(str, pos, caches)):
        case(Success(pos, value)):
          expr2(str, pos, caches)
        case(Failure()):
          Failure()
        end
      end
    end

    def alt(MkExpr(expr1), MkExpr(expr2)):
      MkExpr^(str, pos, caches):
        match(expr1(str, pos, caches)):
        case(Success(pos, value)):
          Success(pos, value)
        case(Failure()):
          expr2(str, pos, caches)
        end
      end
    end

    def zero_or_more(MkExpr(expr)):
      MkExpr^(str, pos, caches):
        def loop(rev_values, pos):
          match(expr(str, pos, caches)):
          case(Success(pos, value)):
            loop([value, *rev_values], pos)
          case(Failure()):
            Success(pos, List::rev(rev_values))
          end
        end
        loop([], pos)
      end
    end

    def not(MkExpr(expr)):
      MkExpr^(str, pos, caches):
        match(expr(str, pos, caches)):
        case(Success(pos, value)):
          Failure()
        case(Failure()):
          Success(pos, ())
        end
      end
    end

    def return(value):
      MkExpr^(str, pos, caches):
        Success(pos, value)
      end
    end

    def bind(MkExpr(expr), proc):
      MkExpr^(str, pos, caches):
        match(expr(str, pos, caches)):
        case(Success(pos, value)):
          let MkExpr(expr) = proc(value)
          expr(str, pos, caches)
        case(Failure()):
          Failure()
        end
      end
    end

    def one_or_more(expr):
      bind(expr)^(value):
        bind(zero_or_more(expr))^(values):
          return([value, *values])
        end
      end
    end

    def zero_or_one(expr):
      alt(expr, no_op)
    end

    def and(expr):
      not(not(expr))
    end

    let String::C#(<-) = def_nt
    let Expr#(&+) = seq
    let Expr#(|+) = alt

    export def_nt, char, string, char_set, nt_ref, fail, no_op
    export seq, alt, zero_or_more, not
    export return, bind, one_or_more, zero_or_one, and
    export String::C#(<-), Expr#(&+), Expr#(|+)
  end

  export Success, Failure, parse, New
end
\end{lstlisting}

\end{document}
