% withpage: ページ番号をつける (著者確認用)
% english: 英語原稿用フォーマット
\documentclass{ipsjprosym}
%\documentclass[withpage,english]{ipsjprosym}

\usepackage[dvips]{graphicx}
\usepackage{latexsym}

\begin{document}

% Title, Author %%%%%%%%%%%%%%%%%%%%%%%%%%%%%%%%%
\title{タイトルをここに}

\affiliate{IPSJ}{情報処理学会}
\affiliate{PROSYM}{プログラミング・シンポジウム幹事団}

\author{情報 太郎}{Joho Taro}{IPSJ}[taro@ipsj.or.jp]
\author{プロシン 花子}{Hiroki MIZUNO}{PROSYM}[hanako@prosym.ipsj.or.jp]

\begin{abstract}
[概要(400字程度)]
本テンプレートは,プログラミング・シンポジウム予稿集に掲載される原稿のた
めのスタイルファイルの使い方を示すものである.著者より提出された原稿は,
ヘッダやページ番号が付加されて,B5サイズにて製本される.そのため,スタイ
ルファイルを使用した原稿は,通常よりも大きな余白がとられ,またページ番号
等がつかない.印刷時の問題を避けるため,最終原稿の提出の際には,フォント
の埋め込みを行ってください.○○○○○○○○○○○○○○○○○○○○○○
○○○○○○○○○○○○○○○○○○○○○○○○○○○○○○○○○○○○
○○○○○○○○○○○○○○○○○○○○○○○○○○○○○○○○○○○○
○○○○○○○○○○○○○○○○○○○○○○○○○○○○○○○○○○○○
○○○○○○○○○○○○○○○○○○○○○○○○○○○○○○○○○○○○
○○○○○○○○○○○○○○○○○○○○○○○○○○○○○○○○○○○○
\end{abstract}

\begin{jkeyword}
プログラミング・シンポジウム,冬,予稿集
\end{jkeyword}

\maketitle

% Body %%%%%%%%%%%%%%%%%%%%%%%%%%%%%%%%%
\section{はじめに}

本テンプレートは「プログラミング・シンポジウム予稿集」に掲載される
原稿のためのクラスファイル(\verb|ipsjprosym.cls|)の使い方について説明するものである.

プログラミング・シンポジウム予稿集の原稿は,印刷前にまとめてページ番号が振られ,
B5版で製本される.本クラスファイルを用いることで,そのような原稿を作成できるはずである.

\section{オプション}

\verb|ipsjprosym.cls| では以下の二つのオプションを提供している.
\begin{itemize}
 \item \verb|withpage|: 著者が執筆上必要な場合のため,ページ番号をつける
 \item \verb|english|: 英語で執筆される場合にフォーマットを調整する.
\end{itemize}

\section{論文1ページ目の情報}

論文の1ページ目には,タイトル,著者名,著者所属,概要,キーワードが配置される.
それぞれ,
\begin{itemize}
\item \verb|\title| 
\item \verb|\author|
\item \verb|affiliate|
\item \verb|\begin{abstract}|〜\verb|\end{abstract}|
\item \verb|\begin{jkeyword}|〜\verb|\end{jkeyword}|
\end{itemize}
によって記述する.
その後,\verb|\maketitle| コマンドによってそれらの情報が配置される.

以下,通常の論文と同様の形式で記述して下さい.

\section{まとめ}

本テンプレートでは,プログラミング・シンポジウム向けの原稿を,
\LaTeX を用いて準備する方法についてごく簡単に示した.

本テンプレートに関する質問・バグ報告は,
第56回プログラミングシンポジウム予稿集担当(松崎公紀)\verb|matsuzaki.kiminori@kochi-tech.ac.jp|
まで連絡下さい.

\begin{acknowledgment}
謝辞が必要であれば,ここに書く.
\end{acknowledgment}

% BibTeX を使用する場合 %%%%%%%%%%%%%%%%%%%%%%%%%%%%%%%%%
% \bibliographystyle{ipsjsort}
% \bibliography{ref}

% BibTeX を使用しない場合
\begin{thebibliography}{9}
 \bibitem{latex} 奥村晴彦, 黒木裕介: \textbf{LaTeX2e美文書作成入門}. 技術評論社, 2013.
\end{thebibliography}

\end{document}
