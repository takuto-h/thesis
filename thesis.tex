%%
% このファイルは、筑波大学情報学群情報科学類の
% 卒業研究論文本体のサンプルです。
% このファイルを書き換えて、この例と同じような書式の論文本体を
% LaTeXを使って作成することができます。
% 
% PC環境や、LaTeX環境の設定によっては漢字コードや改行コードを
% 変更する必要があります。
%%
\documentclass[a4paper,11pt,dvipdfmx]{jreport}

%%【PostScript, JPEG, PNG等の画像の貼り込み】
%% 利用するパッケージを選んでコメントアウトしてください。
\usepackage{graphicx} % for \includegraphics[width=3cm]{sample.eps}
%\usepackage{epsfig} % for \psfig{file=sample.eps,width=3cm}
%\usepackage{epsf} % for \epsfile{file=sample.eps,scale=0.6}
%\usepackage{epsbox} % for \epsfile{file=sample.eps,scale=0.6}

%% dvipdfm を使う場合(dvi->pdfを直接生成する場合)
%\usepackage[dvipdfm]{color,graphicx}
%% dvipdfm を使ってPDFの「しおり」を付ける場合
%\usepackage[dvipdfm,bookmarks=true,bookmarksnumbered=true,bookmarkstype=toc]{hyperref}
%% 参考:dvipdfm 日本語版
%% http://hamilcar.phys.kyushu-u.ac.jp/~hirata/dvipdfm/

\usepackage[bookmarksnumbered=true]{hyperref}
\usepackage{pxjahyper}

\usepackage{times} % use Times Font instead of Computer Modern

\setcounter{tocdepth}{3}
\setcounter{page}{-1}

\setlength{\oddsidemargin}{0.1in}
\setlength{\evensidemargin}{0.1in} 
\setlength{\topmargin}{0in}
\setlength{\textwidth}{6in} 
%\setlength{\textheight}{10.1in}
\setlength{\parskip}{0em}
\setlength{\topsep}{0em}

%\newcommand{\zu}[1]{{\gt \bf 図\ref{#1}}}

%% タイトル生成用パッケージ(重要)
\usepackage{coins-jp}
\usepackage{jumoline}

%% タイトル
%% 【注意】タイトルの最後に\\ を入れるとエラーになります
\title{\Underline{レキシカル環境にメソッドを定義する\\オブジェクト指向言語Suzu}}
%% 著者
\author{林 拓人}
%% 指導教員
\advisor{前田敦司}

%% 専攻名 と 年月 (提出年月)
%% 年月は必要に応じて書き替えてください。
\heiseiyear{26}  % 平成の年度
\majorfield{ソフトウェアサイエンス主専攻}
%\majorfield{情報システム主専攻}
%\majorfield{知能情報メディア主専攻}

\makeatletter%% プリアンブルで定義する場合は必須

%% (j)report・(j)book クラスの場合
%% 
\renewenvironment{thebibliography}[1]% 再定義
{\chapter*{\bibname\@mkboth{\bibname}{\bibname}}%
	\addcontentsline{toc}{chapter}{\numberline{}\bibname}% この行追加
	\list{\@biblabel{\@arabic\c@enumiv}}%
	{\settowidth\labelwidth{\@biblabel{#1}}%
		\leftmargin\labelwidth
		\advance\leftmargin\labelsep
		\@openbib@code
		\usecounter{enumiv}%
		\let\p@enumiv\@empty
		\renewcommand\theenumiv{\@arabic\c@enumiv}}%
	\sloppy
	\clubpenalty4000
	\@clubpenalty\clubpenalty
	\widowpenalty4000%
	\sfcode`\.\@m}
{\def\@noitemerr
	{\@latex@warning{Empty `thebibliography' environment}}%
	\endlist}

\makeatother%% プリアンブルで定義する場合は必須

\usepackage{listings}

\begin{document}
\maketitle
\thispagestyle{empty}
\newpage

\thispagestyle{empty}
\vspace*{20pt plus 1fil}
\parindent=1zw
\noindent
%%
%% 論文の概要(Abstract)
%%
\begin{center}
{\Large \bf 要  旨}
\vspace{2cm}
\end{center}

従来のオブジェクト指向言語は変数をローカル定義できるのに対しメソッドを
ローカル定義することはできない.
ローカル変数の有用性は広く認められており,ローカルメソッドを定義できれば
これも有用であると考えられる.

本研究はローカルメソッドを定義可能なオブジェクト指向言語Suzuを開発し
その有用性を実証する.
Suzuではローカルメソッドを用いてメソッドを局所的に追加・再定義することで,
グローバル環境を汚染しない可読性の高い内部DSLを構築できる.
またローカルメソッドの実現にあたり変数定義とメソッド定義のシンタックスおよび
セマンティクスを統一したことで,変数を扱うのと同じモジュールシステムにより
メソッドを柔軟にグループ化・共通化・再利用できる.

従来の言語の類似する機構と比較した利点は,メソッドのスコープをブロックという
細かい単位で制御できることと,モジュールによる衝突回避が利用できることである.

%%%%%
\par
\vspace{0pt plus 1fil}
\newpage

\pagenumbering{roman} % I, II, III, IV 
\tableofcontents
%\listoffigures
%\listoftables

\pagebreak \setcounter{page}{1}
\pagenumbering{arabic} % 1,2,3


\chapter{序論}

オブジェクト指向言語とは,オブジェクト指向プログラミングの支援機構を持つ
プログラミング言語の総称である.
多くのオブジェクト指向言語が持つ概念としてクラスとメソッドが挙げられる.
クラスはオブジェクトの属性,メソッドはオブジェクトに対する操作である.
あるクラスを属性として持つオブジェクトをそのクラスのインスタンスと呼ぶ.
オブジェクトに対しメソッド名を指定してメソッド呼び出しを行うと,
オブジェクトのクラスに応じて適切なメソッドが選択され呼び出される.

オブジェクト指向言語に限らずほとんどのプログラミング言語ではローカル変数を
定義できる.
ローカル変数とは定義したブロックや関数内でのみ参照可能な変数のことである.
変数を参照可能な範囲のことをその変数のスコープと呼ぶ.
ローカル変数はスコープがブロックや関数単位で細かく区切られており,
そのブロックや関数内でのみ使いたい変数を新たに定義したり,外側のブロックや関数で
定義された変数を局所的に再定義したりすることができる.
ほとんどのプログラミング言語においてローカル変数を定義できることは,
これらローカル変数の有用性が広く認められている証拠である.

従来のオブジェクト指向言語は,メソッドを変数のようにローカル定義することはできない.
クラス定義の内側でメソッド定義を行う言語においてメソッドのスコープはあくまでクラス単位
でのみ制御でき,ブロックや関数単位で局所的にメソッドを定義することは不可能である.
しかしながらローカル変数が有用であるように,定義したブロックや関数内でのみ参照可能な
ローカルメソッドを定義できれば,メソッドを局所的に追加・再定義することができ
有用であると考えられる.

そこで,本研究はローカルメソッドを定義可能なオブジェクト指向言語Suzuを開発し,
その有用性を実証する.
ローカルメソッドを実現するにあたっては,変数定義とメソッド定義のシンタックスおよび
セマンティクスを統一するというアプローチをとる.
これらが統一されれば必然的にローカル変数に対応するローカルメソッドを定義できる.
変数とメソッドの違いは識別に必要な識別子の数である.
変数は変数名1つによって識別されるのに対し,メソッドはクラス名とメソッド名の2つに
よって識別される.
ここでクラス名とメソッド名の2つを組にして1つの識別子としてとらえれば,変数名を用いる
代わりにクラス名とメソッド名の組を用いることで変数と同じシンタックスおよびセマンティクスで
メソッドを定義することができる.

ローカルメソッドの活用法として内部DSLの構築が挙げられる.
DSLとはDomain Specific Languageの略で,ある領域の問題を解決することに
特化した言語のことである.
特に内部DSLとは,他の言語のプログラム内に記述できるDSLのことを指す.
Suzuは演算子の適用をメソッド呼び出しとして解釈するため,対応するメソッドを定義
することでユーザーが自由に演算子を追加・再定義できる.
演算子をローカルメソッドとして定義すれば,演算子を用いた可読性の高い内部DSLを
メソッド衝突のリスクを避けつつ利用することができる.

またSuzuでは変数定義とメソッド定義のシンタックスおよびセマンティクスを統一したことで,
変数とメソッドを同じモジュールシステムによって統一的に扱うことができる.
Suzuにはクラスを継承する機能が存在しないが,このモジュールシステムを用いることで
多重継承に相当するメソッドの共通化・再利用が可能である.

従来の言語においてSuzuと似た柔軟性を持つ機構は存在する.
それらと比較した時のSuzuの利点は,メソッドのスコープをブロックという細かい単位で
制御できることと,モジュールシステムによる衝突回避が利用できることである.

本稿は次のような構成をとる.
第\ref{chapter:proposal}章ではローカルメソッドを実現するために
変数定義とメソッド定義のシンタックスおよびセマンティクスを統一する手法を提案する.
第\ref{chapter:implementation}章ではこの手法を適用し設計した言語の実例として
ローカルメソッドを定義可能なプログラミング言語Suzuの言語仕様を解説する.
第\ref{chapter:discussion}章ではローカルメソッドなどの特徴を生かしたSuzuの
プログラム例によってその有用性を検証する.
第\ref{chapter:related-work}章ではSuzuと似た柔軟性を持つ従来の機構との
関連性について述べる.
第\ref{chapter:future-work}章でSuzuの仕様と実装にまつわる課題を述べ,
第\ref{chapter:conclusion}章で結論を述べる.


\chapter{提案手法}
\label{chapter:proposal}

従来のオブジェクト指向言語は変数をローカル定義することはできても
メソッドをローカル定義することはできない.
本研究はメソッドのローカル定義を可能にしその有用性を検証するため,
変数定義とメソッド定義のシンタックスおよびセマンティクスを統一する手法と,
これを適用したことでメソッドをローカル定義できる新しいオブジェクト指向言語を
提案する.

\section{従来のオブジェクト指向言語}

多くのオブジェクト指向言語は定義を記述したブロックや関数・メソッド内でのみ
参照可能なローカル変数を定義することができる.
例えばC++やJavaでは
\begin{quote}
	\begin{verbatim}
	{
	  int v = ...
	  ...
	}
	\end{verbatim}
\end{quote}
のようにすれば,ブロック\verb|{|~\verb|}|内でのみ参照可能な
変数\verb|v|を定義できる.
Rubyでは
\begin{quote}
	\begin{verbatim}
	1.times do
	  v = ...
	  ...
	end
	\end{verbatim}
\end{quote}
のようにすれば,ブロック\verb|do|~\verb|end|内でのみ参照可能な
変数\verb|v|を定義でき,
\begin{quote}
	\begin{verbatim}
	def m
	  v = ...
	  ...
	end
	\end{verbatim}
\end{quote}
のようにすれば,メソッド\verb|m|内でのみ参照可能な変数\verb|v|を
定義できる.
Pythonでは
\begin{quote}
	\begin{verbatim}
	def f():
	  v = ...
	  ...
	\end{verbatim}
\end{quote}
のようにすれば,関数\verb|f|内でのみ参照可能な変数\verb|v|を
定義できる.

ローカル変数はブロックや関数・メソッド単位で局所的に変数を追加・再定義
することができ有用である.
オブジェクト指向言語に限らずほとんどのプログラミング言語でローカル変数を
定義できることはこの有用性が広く認められている証拠だといえる.
これに対しメソッドのローカル定義をサポートするオブジェクト指向言語は
著者の知る限り存在せず,その有用性が検証されているとは言えない.
しかしながらローカル変数が有用であるように,ローカル変数に対応する
ローカルメソッドを定義できれば,ブロックや関数・メソッド単位で局所的に
メソッドを追加・再定義でき有用だと考えられる.

変数とメソッドの根本的な違いは識別に必要な識別子の数である.
メソッドを定義する際にはメソッドの内容の他にクラス名とメソッド名の2つを
指定する必要がある.
例えばクラス\verb|C|のインスタンスを操作するメソッド\verb|m|を
定義するには,C++では
\begin{quote}
	\begin{verbatim}
	class C {
	public:
	  void m(){
	    ...
	  }
	};
	\end{verbatim}
\end{quote}
Javaでは
\begin{quote}
	\begin{verbatim}
	class C {
	  void m(){
	    ...
	  }
	}
	\end{verbatim}
\end{quote}
Rubyでは
\begin{quote}
	\begin{verbatim}
	class C
	  def m
	    ...
	  end
	end
	\end{verbatim}
\end{quote}
Pythonでは
\begin{quote}
	\begin{verbatim}
	class C:
	  def m(self):
	    ...
	\end{verbatim}
\end{quote}
のようにする.
どの言語においてもメソッドを定義するにはクラス名\verb|C|とメソッド名\verb|m|
の2つが必要である.
変数を識別する識別子が変数名1つであるのに対し,メソッドを識別する識別子は
クラス名とメソッド名の2つであると言える.


\section{提案するオブジェクト指向言語}

従来のオブジェクト指向言語はメソッドをローカル定義できないが,ローカル変数の
有用性からローカルメソッドも有用であると考えられる.
そこで本研究はローカルメソッドが定義可能なオブジェクト指向言語Suzuを開発し,
その有用性を検証する.

変数とメソッドの違いは識別に必要な識別子の数であることはすでに述べた.
変数は変数名1つによって識別されるのに対し,メソッドはクラス名とメソッド名の2つに
よって識別される.
ここでクラス名とメソッド名の2つを組にして1つの識別子としてとらえれば,変数名を用いる
代わりにクラス名とメソッド名の組を用いることで変数と同じシンタックスおよびセマンティクスで
メソッドを定義できる.
例えば,
\begin{quote}
	\begin{verbatim}
	let v = ...
	\end{verbatim}
\end{quote}
のようにして変数を定義し,
\begin{quote}
	\begin{verbatim}
	let C#m = ...
	\end{verbatim}
\end{quote}
のようにしてメソッドを定義する.\verb|C#m|はクラス名\verb|C|とメソッド名\verb|m|の
組である.
これにより,\verb|begin|~\verb|end|をブロックとすると,
\begin{quote}
	\begin{verbatim}
	begin:
	  let v = ...
	  ...
	end
	\end{verbatim}
\end{quote}
のようにしてブロックローカルな変数を定義でき,
\begin{quote}
	\begin{verbatim}
	begin:
	  let C#m = ...
	  ...
	end
	\end{verbatim}
\end{quote}
のようにしてブロックローカルなメソッドを定義できる.

ローカル変数と同じようにローカルメソッドを定義できれば,ブロックや関数・メソッド単位で
局所的にメソッドを追加・再定義でき有用だと考えられる.
第\ref{chapter:implementation}章ではこのようにクラス名とメソッド名の組を1つの
識別子としてとらえてメソッドを定義することで変数定義とメソッド定義のシンタックスおよび
セマンティクスを統一し,
ローカル変数に対応するローカルメソッドを定義可能にしたオブジェクト指向言語Suzuの
言語仕様を解説する.
第\ref{chapter:discussion}章ではローカルメソッドや変数とメソッドの統一的な
シンタックスおよびセマンティクスを生かしたSuzuのプログラム例によってその有用性を検証する.


\chapter{プログラミング言語Suzu}
\label{chapter:implementation}

ローカル変数に対応するローカルメソッドが定義可能なプログラミング言語Suzuの
言語仕様をコード例を交えて解説する.
Suzuはクラス名とメソッド名の組を用いて変数と同じようにメソッドを定義することで,
変数定義とメソッド定義のシンタックスおよびセマンティクスを統一している.
コード例において式を省略する際には\verb|...|を用いる.

\section{基本的な要素}

変数は\verb|let|を用いて定義する.例えば変数\verb|v|を\verb|...|の値で
初期化して定義するには以下のようにする.
\begin{quote}
\begin{verbatim}
let v = ...
\end{verbatim}
\end{quote}

Suzuでは関数リテラルを記述できる.引数\verb|param1|,\verb|param2|を受け取って
\verb|...|を実行する関数は以下のように書ける.
\begin{quote}
\begin{verbatim}
^(param1, param2){ ... }
\end{verbatim}
\end{quote}

関数は\verb|def|を用いて定義できる.
\begin{quote}
\begin{verbatim}
def func(param1, param2):
  ...
end
\end{verbatim}
\end{quote}
これは以下のコードと等価である.
\begin{quote}
\begin{verbatim}
let func = ^(param1, param2){ ... }
\end{verbatim}
\end{quote}

関数\verb|func|を引数\verb|arg1|,\verb|arg2|で呼び出すには以下のようにする.
\begin{quote}
\begin{verbatim}
func(arg1, arg2)
\end{verbatim}
\end{quote}

最後の引数として関数リテラルを渡す際は専用の構文が使える.例えば
\begin{quote}
\begin{verbatim}
func(arg1, arg2, ^(param1, param2){ ... })
\end{verbatim}
\end{quote}
という関数呼び出しは
\begin{quote}
\begin{verbatim}
func(arg1, arg2)^(param1, param2):
  ...
end
\end{verbatim}
\end{quote}
と書ける.他の実引数や関数リテラルの仮引数,またはその両方がない場合,
\begin{quote}
\begin{verbatim}
func^(param1, param2):
  ...
end

func(arg1, arg2):
  ...
end

func:
  ...
end
\end{verbatim}
\end{quote}
のように省略して書ける.

Suzuでは\verb|:|からインデントの終わりまで(または\verb|{|から\verb|}|まで)を
ブロックと呼ぶ.
ブロック内のコードを実行する際は新たなレキシカル環境\cite{DesignConceptsInPL}
が生成され有効になり,実行が終わると破棄される.
変数は有効なレキシカル環境に定義される.よってブロック内で定義された変数は
ブロック内でのみ参照可能なローカル変数となる.例えば
\begin{quote}
\begin{verbatim}
def begin(thunk):
  thunk()
end
\end{verbatim}
\end{quote}
のような無引数の関数\verb|thunk|を受け取って呼び出す高階関数
\verb|begin|に
\begin{quote}
\begin{verbatim}
begin:
  let v = ...
  ...
end
\end{verbatim}
\end{quote}
のように無引数の関数リテラルを渡して呼び出させた場合,変数\verb|v|は
\verb|begin|から\verb|end|の間のみで有効なローカル変数となる.

\section{オブジェクトシステム}

Suzuでは変数名の代わりにクラス名とメソッド名の組を指定し,メソッドを変数と同様
レキシカル環境に定義する.
つまりメソッドを定義するには,
\begin{quote}
\begin{verbatim}
def C#m(self, param1, param2):
  ...
end
\end{verbatim}
\end{quote}
または
\begin{quote}
\begin{verbatim}
let C#m = ^(self, param1, param2){ ... }
\end{verbatim}
\end{quote}
のようにする.\verb|C|はクラス名,\verb|m|はメソッド名,\verb|C#m|は
クラス名とメソッド名の組である.

定義したメソッドは\verb|C#m|と書くことで変数のように参照できる他,
\verb|inst|をクラス\verb|C|のインスタンスとすると以下のようにして呼び出せる.
\begin{quote}
\begin{verbatim}
inst.m(arg1, arg2)
\end{verbatim}
\end{quote}
これは以下の呼び出しと同じ結果となる.
\begin{quote}
\begin{verbatim}
C#m(inst, arg1, arg2)
\end{verbatim}
\end{quote}
このように,メソッドの第1引数にはメソッド呼び出しの対象となったオブジェクト自身が,
第2引数以降には実引数が渡される.
つまり\verb|self|には\verb|inst|,\verb|param1|には\verb|arg1|,
\verb|param2|には\verb|arg2|の内容がそれぞれ代入される.

クラスは以下のようにして定義する.
\begin{quote}
\begin{verbatim}
class Point = make_point:
  x
  y
end
\end{verbatim}
\end{quote}
これにより,クラス\verb|Point|,コンストラクタ関数\verb|make_point|,
ゲッターメソッド\verb|Point#x|,\verb|Point#y|が定義される.
\verb|make_point|は\verb|x|と\verb|y|というフィールドを持った
\verb|Point|のインスタンスを生成する関数である.
フィールドへのアクセスは\verb|Point#x|,\verb|Point#y|を介してのみ行える.
例えば
\begin{quote}
\begin{verbatim}
let p = make_point(1, 2)
\end{verbatim}
\end{quote}
とすると,\verb|p.x|は\verb|1|,\verb|p.y|は\verb|2|となる.

クラス定義の際,
\begin{quote}
\begin{verbatim}
class Point = make_point:
  mutable x
  mutable y
end
\end{verbatim}
\end{quote}
のようにフィールド名の前に\verb|mutable|と書くと,上記に加えて
セッターメソッド\verb|Point#(x=)|,\verb|Point#(y=)|が定義される.
\begin{quote}
\begin{verbatim}
p.x = 3
p.y = 4
\end{verbatim}
\end{quote}
のような式は
\begin{quote}
\begin{verbatim}
p.(x=)(3)
p.(y=)(4)
\end{verbatim}
\end{quote}
というメソッド呼び出しに解釈されるため,これを用いて\verb|Point#(x=)|や
\verb|Point#(y=)|を呼び出すことで\verb|p|のフィールド\verb|x|,\verb|y|を
書き換えられる.

Suzuでは演算子の適用もメソッド呼び出しとして解釈される.
正規表現\verb$[-+*/%&|=<>]+$にマッチする文字列は演算子として扱われ,
優先順位や結合則は演算子の1文字目に依存する.
例えば
\begin{quote}
\begin{verbatim}
x <- y
\end{verbatim}
\end{quote}
という式は
\begin{quote}
\begin{verbatim}
x.(<-)(y)
\end{verbatim}
\end{quote}
と解釈される.優先順位や結合則は\verb|<|と同じである.

ここまで述べてきたことから分かるように,Suzuではクラス定義とメソッド定義を
分けて記述する.
つまりはクラスを定義した後に,そのクラスのインスタンスを操作するメソッドを
自由に追加できる.
また,メソッドは変数と同様レキシカル環境に定義されるため,特定のブロック内でのみ
有効なローカルメソッドを定義できる.
さらに演算子の適用もメソッド呼び出しとして解釈されるため,ローカルメソッドを
定義することでクラスごとの演算子の意味を局所的に変えることもできる.

\section{モジュールシステム}

モジュールとは,変数や関数などの複数の定義をまとめ,指定した名前の変数や
関数のみを限定的に公開(エクスポート)する機能である.
エクスポートされていない名前はモジュールの外から見えないため,
実装の詳細を隠蔽したり,名前の衝突を避けたりするのに有用である.

Suzuはメソッドを変数と同様レキシカル環境に定義するため,モジュールによって
変数と同じようにメソッドの可視性を制御できる.
モジュールを定義するには以下のようにする.
\begin{quote}
\begin{verbatim}
module M:
  let f = ...
  let C#m = ...
  ...
  export f, C#m
end
\end{verbatim}
\end{quote}
これにより,変数\verb|f|とメソッド\verb|C#m|をエクスポートするモジュール
\verb|M|が定義される.
\verb|let|(または\verb|def|)を用いて変数やメソッドを定義した後,
\verb|export|の後にエクスポートしたい変数の変数名や,メソッドを表すクラス名と
メソッド名の組を並べて指定する.
エクスポートされている変数やメソッドは\verb|::|を用いて,
\verb|M::f|,\verb|M::(C#m)|のように参照できる.
モジュール内で定義した変数やメソッドをエクスポートしなかった場合,
それらはモジュール内でのみ有効なモジュールローカルな変数,またはメソッドとなる.

モジュールは任意のスコープで\verb|open|できる.例えば
\begin{quote}
\begin{verbatim}
begin:
  open M
  ...
end
\end{verbatim}
\end{quote}
とすると,\verb|begin|から\verb|end|までのブロックで\verb|M|から
エクスポートされている変数やメソッドを修飾子なしで参照できるようになる.
\verb|M|を先ほど定義したものとすると,\verb|f|,\verb|C#m|のように
変数やメソッドを参照可能となる.

既存のモジュールを活用して新しいモジュールを定義する際は
\verb|include|が便利である.
\begin{quote}
\begin{verbatim}
module B:
  include A
  ...
end
\end{verbatim}
\end{quote}
この例ではモジュール\verb|B|内でモジュール\verb|A|を\verb|include|している.
\verb|include A|とするとまず\verb|open A|が行われ,次に\verb|A|から
エクスポートされている
変数やメソッドをすべて\verb|B|からもエクスポートする.
これによりモジュール\verb|A|を拡張した新たなモジュール\verb|B|を容易に定義できる.

\verb|open|や\verb|include|によって新しく参照可能になる変数やメソッドが
衝突する場合,そのコードはエラーとなる.
例えば以下のコードは\verb|open Y|で\verb|C#q|が衝突するためエラーとなる.
\begin{quote}
\begin{verbatim}
module X:
  ... 
  export C#p, C#q
end
module Y:
  ...
  export C#q, C#r
end
begin:
  open X
  open Y
  ...
end
\end{verbatim}
\end{quote}
これを回避するには\verb|except|を用いる.
\begin{quote}
\begin{verbatim}
begin:
  open X except C#q
  open Y
  ...
end
\end{verbatim}
\end{quote}
この例では\verb|X|を\verb|open|する際\verb|C#q|を\verb|except|している.
これによりこの時点で\verb|C#q|はインポートされない.
よって\verb|open Y|としても衝突は起こらずエラーは回避される.

Suzuにはパラメータ化されたモジュールとしてのトレイトという機構がある.
トレイトは値を受け取ってモジュールを返す関数である.
\begin{quote}
\begin{verbatim}
trait T(param1, param2):
  let f = ...
  let C#m = ...
  ...
  export f, C#m
end
\end{verbatim}
\end{quote}
この例では\verb|param1|と\verb|param2|を受け取って,\verb|f|と\verb|C#m|を
エクスポートするモジュールを返すトレイト\verb|T|を定義している.
トレイトの活用法については第\ref{chapter:discussion}章で述べる.


\chapter{考察}
\label{chapter:discussion}

Suzuのローカルメソッドやモジュールシステムなどを生かしたプログラム例によって,
変数定義とメソッド定義のシンタックスおよびセマンティクスを統一することにより
もたらされる有用性を検証する.
なお,\verb|p|は引数として受け取った値を出力する関数,
\verb|//|から行末まではコメントである.
本章ではコメントとして\verb|//=>|の後にプログラムの出力内容を記述する.

\section{組み込みオブジェクトに対するメソッドの追加}
\label{section:builtin}

Suzuではメソッド定義がクラス定義から独立しているため,組み込みオブジェクトに
対するメソッドもユーザーが自由に追加できる.例えば,
\begin{quote}
\begin{verbatim}
p("program".pluralize)    //=> "programs"
p("programs".singularize) //=> "program"
p("person".pluralize)     //=> "people"
p("people".singularize)   //=> "person"
\end{verbatim}
\end{quote}
のように,文字列を複数形および単数形に変換するメソッド
\verb|pluralize|,\verb|singularize|を追加したければ,
\begin{quote}
\begin{verbatim}
def String::C#pluralize(self):
  ...
end
def String::C#singularize(self):
  ...
end
\end{verbatim}
\end{quote}
のようにして定義できる.
\verb|String::C|というのはモジュール\verb|String|の変数\verb|C|で,
組み込みオブジェクトである文字列のクラス名を指している.

定義をブロック内で行えばこれらはローカルメソッドとなるため,メソッドの衝突を
未然に防ぐことができる.
\begin{quote}
\begin{verbatim}
begin:
  def String::C#pluralize(self):
    ...
  end
  def String::C#singularize(self):
    ...
  end
  ...
end
\end{verbatim}
\end{quote}
上の例では2つのメソッド定義は\verb|begin|から\verb|end|の間でのみ
有効である.

また,複数のメソッドをまとめてモジュールとして提供することもできる.
\begin{quote}
\begin{verbatim}
module Noun:
  def String::C#pluralize(self):
    ...
  end
  def String::C#singularize(self):
    ...
  end
  ...
  export String::C#pluralize, String::C#singularize
end
\end{verbatim}
\end{quote}
これをブロック内で\verb|open|すれば直接定義したのと同じ効果が得られる.
\begin{quote}
\begin{verbatim}
begin:
  open Noun
  ...
end
\end{verbatim}
\end{quote}
このように,利便性の高いメソッド群をユーザーがモジュールとして提供し,
局所的に有効にすることでメソッド衝突の危険を避けながら利用できる.


\section{演算子の局所的な再定義}
\label{section:operator}

Suzuでは演算子の適用がメソッド呼び出しとして扱われる.そのため演算子に対応する
メソッドを局所的に再定義することで,その振る舞いを変えることができる.

例えば整数同士の割り算を行うメソッド\verb|Int::C#(/)|は,組み込みのものは
整数を返す.
これを浮動小数点数を返す関数\verb|Int::quo|で置き換えたい場合,
単に\verb|let|を用いればよい.
\begin{quote}
\begin{verbatim}
p(3 / 2) //=> 1
begin:
  let Int::C#(/) = Int::quo
  p(3 / 2) //=> 1.5
end
p(3 / 2) //=> 1
\end{verbatim}
\end{quote}

また以下のようなモジュール\verb|Quotient|を定義しておけば,
ブロック内で\verb|open|することでそのスコープでのみ浮動小数点数を
返すよう変えることができる.
\begin{quote}
\begin{verbatim}
module Quotient:
  let Int::C#(/) = Int::quo
  export Int::C#(/)
end

p(3 / 2) //=> 1
begin:
  open Quotient
  p(3 / 2) //=> 1.5
end
p(3 / 2) //=> 1
\end{verbatim}
\end{quote}

Suzuのメソッドは変数と同じレキシカルスコープなので,ブロック内から呼び出した先の
関数に影響を与えることなく安全に演算子の再定義ができる.


\section{内部DSL}
\label{section:dsl}

DSLとはDomain Specific Languageの略で,特定の問題を解決するための
ミニ言語のことである.
特に内部DSLとは,プログラム内に記述できる式として構築したDSLを指す.

Suzuは演算子として使用できる文字列に自由度があるため,
演算子をローカルメソッドとして定義すれば,グローバル環境を汚染せず可読性の高い
内部DSLを作成できる.
以下の例は演算子式によって正規表現を構築できるDSLを提供するモジュール
\verb|PrettyRegex|を定義している.\verb|String::format|は
文字列整形関数である.
\begin{quote}
\begin{verbatim}
module PrettyRegex:
  def String::C#(|)(lhs, rhs):
    String::format("({0}|{1})", lhs, rhs)
  end
  def String::C#(+)(lhs, rhs):
    String::format("{0}{1}", lhs, rhs)
  end
  def Char::C#(-)(lhs, rhs):
    String::format("[{0}-{1}]", lhs, rhs)
  end
  def String::C#one_or_more(exp):
    String::format("{0}+", exp)
  end
  export String::C#(|)
  export String::C#(+)
  export String::C#one_or_more
  export Char::C#(-)
end
\end{verbatim}
\end{quote}
これは以下のように使用できる.
\begin{quote}
\begin{verbatim}
let regex = begin:
  open PrettyRegex
  ("foo"|"bar")+('0'-'9').one_or_more
end
p(regex) //=> "(foo|bar)[0-9]+"
\end{verbatim}
\end{quote}
縦棒やプラス,ハイフンといった衝突の危険が高いと思われる短い演算子も,
スコープを限定して定義することで安心して使用できる.


\section{メソッドの柔軟なグループ化}
\label{section:grouping}

Suzuのメソッド定義はクラス定義とは独立しており,モジュールに所属する形で
定義することができる.
そのため対象のクラスによらず関連性の高いメソッドを集めて柔軟にグループ化できる.

例えばオブジェクトをJSON形式でシリアライズするメソッド\verb|to_json|を
定義したい場合,クラスにメソッドを定義する言語では定義が各クラスに分散して
しまう.

Suzuでは以下のように複数のクラスのインスタンスに対するメソッドを提供する
モジュールを定義することで,関連するメソッドを1箇所にまとめて定義できる.
\begin{quote}
	\begin{verbatim}
	module ToJSON:
	  let    Int::C#to_json = ...
	  let String::C#to_json = ...
	  let   List::C#to_json = ...
	  ...
	end
	\end{verbatim}
\end{quote}
利用する側は有効にしたいスコープで\verb|open ToJSON|とすればよい.

また,ユーザーが新たに定義したデータ型などさらに多くのオブジェクトにメソッドを
提供したい場合,
\begin{quote}
	\begin{verbatim}
	module ToJSONExt:
	  include ToJSON
	  let UserDefinedData1#to_json = ...
	  let UserDefinedData2#to_json = ..
	  ...
	end
	\end{verbatim}
\end{quote}
のように\verb|include|を用いて,既存のモジュールを拡張した
新たなモジュールを定義できる.


\section{トレイトによるメソッドの共通化}
\label{section:traits}

複数のクラスに同じ内容のメソッドを定義したい場合,トレイトによってメソッドの定義を
共通化できる.

例えば銀行口座を表すクラス\verb|BankAccount|と株式口座を表すクラス
\verb|StockAccount|にメソッドを定義することを考えてみる.
\verb|BankAccount|は残高を表すフィールド\verb|balance|を持つ.
\begin{quote}
	\begin{verbatim}
	class BankAccount = make_bank_account:
	  mutable balance
	end
	\end{verbatim}
\end{quote}
預け入れを行うメソッド\verb|deposit!|と引き出しを行うメソッド
\verb|withdraw!|は以下のように書ける.
\begin{quote}
	\begin{verbatim}
	def BankAccount#deposit!(self, x):
	  self.balance = self.balance + x
	end
	
	def BankAccount#withdraw!(self, x):
	  self.balance = self.balance - x
	  if(self.balance < 0):
	    self.balance = 0
	  end
	end
	\end{verbatim}
\end{quote}
フィールドの参照およびフィールドへの代入には,クラス定義によって自動的に定義された
ゲッターメソッド\verb|balance|とセッターメソッド\verb|balance=|を使用している.

これに対し\verb|StockAccount|は株数を表すフィールド\verb|num_shares|と
株単価を表すフィールド\verb|price_per_share|を持つ.
\begin{quote}
	\begin{verbatim}
	class StockAccount = make_stock_account:
	  mutable num_shares
	  price_per_share
	end
	\end{verbatim}
\end{quote}
残高を取得するメソッド\verb|balance|と残高を変更するメソッド\verb|balance=|
は以下のように定義できる.
\begin{quote}
	\begin{verbatim}
	def StockAccount#balance(self):
	  self.num_shares * self.price_per_share
	end
	
	def StockAccount#(balance=)(self, x):
	  self.num_shares = x / self.price_per_share
	end
	\end{verbatim}
\end{quote}
預け入れを行うメソッド\verb|deposit!|と引き出しを行うメソッド
\verb|withdraw!|は以下のように書ける.
\begin{quote}
	\begin{verbatim}
	def StockAccount#deposit!(self, x):
	  self.balance = self.balance + x
	end
	
	def StockAccount#withdraw!(self, x):
	  self.balance = self.balance - x
	  if(self.balance < 0):
	    self.balance = 0
	  end
	end
	\end{verbatim}
\end{quote}

ここで,メソッド\verb|deposit!|および\verb|withdraw!|の内容は,
\verb|BankAccount|と\verb|StockAccount|とで同一であることが分かる.
違いは定義対象のクラスと,使用している\verb|balance|および
\verb|balance=|の定義である.

Suzuではこのようなメソッドの定義をトレイトによって共通化できる.
\begin{quote}
	\begin{verbatim}
	trait Account(C, C#balance, C#(balance=)):
	  def C#deposit!(self, x):
	    self.balance = self.balance + x
	  end
	
	  def C#withdraw!(self, x):
	    self.balance = self.balance - x
	    if(self.balance < 0):
	      self.balance = 0
	    end
	  end
	
	  export C#deposit!, C#withdraw!
	end
	\end{verbatim}
\end{quote}
トレイト\verb|Account|は,クラス\verb|C|,メソッド\verb|C#balance|,
\verb|C#(balance=)|を引数として受け取り,
メソッド\verb|C#deposit!|,\verb|C#withdraw!|が定義された
モジュールを返す.
\verb|Account|に\verb|BankAccount|と\verb|StockAccount|それぞれの
クラスと\verb|balance|および\verb|balance=|を渡して呼び出し戻り値を
\verb|open|すれば,それらを使用した\verb|deposit!|および
\verb|withdraw!|が定義される.
つまり,
\begin{quote}
	\begin{verbatim}
	open Account(BankAccount,
	             BankAccount#balance,
	             BankAccount#(balance=))
	\end{verbatim}
\end{quote}
とすれば\verb|BankAccount#deposit!|と\verb|BankAccount#withdraw!|が
定義され,
\begin{quote}
	\begin{verbatim}
	open Account(StockAccount,
	             StockAccount#balance,
	             StockAccount#(balance=))
	\end{verbatim}
\end{quote}
とすれば\verb|StockAccount#deposit!|と\verb|StockAccount#withdraw!|が
定義される

このように,Suzuではクラスとメソッドによってパラメータ化したモジュールとしてトレイトを
定義することでメソッドの定義を共通化することができ,その呼び出し結果を\verb|open|
することで同じ内容のメソッドを繰り返し書く必要がなくなる.
Suzuには継承機構が存在しないが,メソッドがモジュール等のレキシカル環境に所属するため
実装の継承をモジュールシステムの機能によって代替できることを示している.


\section{多重継承の代替}
\label{section:collision}

Suzuには継承という名の機能は存在しない.しかしながらトレイトを用いれば,多重継承に
相当するようなメソッドの再利用ができる.

コレクションに対しシーケンシャルアクセスを行うための3つのクラス,\verb|ReadStream|,
\verb|WriteStream|,\verb|ReadWriteStream|を実装したいとする.
\verb|ReadStream|は\verb|read|と\verb|seek|,
\verb|WriteStream|は\verb|write|と\verb|seek|,
\verb|ReadWriteStream|は\verb|read|と\verb|write|と\verb|seek|を
それぞれ可能にしたい.
\verb|read|はコレクションの現在位置から1つの要素を読み出し,現在位置を進める.
\verb|write|はコレクションの現在位置に1つのアイテムを書き込み,現在位置を進める.
\verb|seek|は現在位置を変更する.

多重継承を持つ言語ならばまず\verb|ReadStream|と\verb|WriteStream|を定義し,
その後これら2つを継承した\verb|ReadWriteStream|を定義するだろう.
Suzuでは同等のことをトレイトを用いて実現できる.

まず,クラスに\verb|read|と\verb|seek|を提供するトレイト\verb|Readable|と,
クラスに\verb|write|と\verb|seek|を提供するトレイト\verb|Writable|を
定義する.
\begin{quote}
	\begin{verbatim}
	trait Readable(C, C#collection,
	               C#position, C#(position=)):
	  def C#read(self):
	    let elem = self.collection[self.position]
	    self.position = self.position + 1
	    elem
	  end
	  
	  def C#seek(self, position):
	    self.position = position
	  end
	  
	  export C#read, C#seek
	end
	
	trait Writable(C, C#collection,
	               C#position, C#(position=)):
	  def C#write(self, item):
	    self.collection[self.position] = item
	    self.position = self.position + 1
	  end
	  
	  def C#seek(self, position):
	    self.position = position
	  end
	
	  export C#write, C#seek
	end
	\end{verbatim}
\end{quote}
\verb|Readable|と\verb|Writable|はともに,引数としてクラス\verb|C|,
メソッド\verb|collection|,\verb|position|,\verb|position=|を
要求している.

その後,クラス\verb|ReadStream|を定義して\verb|Readable|を適用,
クラス\verb|WriteStream|を定義して\verb|Writable|を適用,
クラス\verb|ReadWriteStream|を定義して\verb|Readable|と
\verb|Writable|を適用する.
\verb|ReadWriteStream|に\verb|Readable|と\verb|Writable|を
適用する際は,\verb|seek|が重複しないようどちらかの定義を\verb|except|
してやる必要がある.
\begin{quote}
	\begin{verbatim}
	class ReadStream = make_read_stream:
	  mutable collection
	  mutable position
	end
	open Readable(
	  ReadStream, ReadStream#collection,
	  ReadStream#position, ReadStream#(position=),
	  )
	
	class WriteStream = make_write_stream:
	  mutable collection
	  mutable position
	end
	open Writable(
	  WriteStream, WriteStream#collection,
	  WriteStream#position, WriteStream#(position=),
	  )
	
	class ReadWriteStream = make_read_write_stream:
	  mutable collection
	  mutable position
	end
	open Readable(
	  ReadWriteStream, ReadWriteStream#collection,
	  ReadWriteStream#position, ReadWriteStream#(position=),
	  )
	open Writable(
	  ReadWriteStream, ReadWriteStream#collection,
	  ReadWriteStream#position, ReadWriteStream#(position=),
	  )
	  except ReadWriteStream#seek
	\end{verbatim}
\end{quote}
このように,Suzuではクラスにメソッドを提供する機能としてトレイトを用いることで,
モジュールの生成と\verb|open|によるメソッドの再利用を行える.
メソッドが重複する際にはモジュールに対する操作である\verb|except|を
使い,衝突を適切に回避できる.


\section{本章のまとめ}

Suzuではメソッド定義がクラス定義から独立しているので,組み込みクラスに対しメソッドを
自由に追加・再定義できる.
メソッドはレキシカル環境に定義されるので,追加や再定義の影響範囲はグローバルにも
ローカルにもできる.
ローカルな影響範囲はブロックという細かい単位で指定できる
(\ref{section:builtin}節).
演算子の適用はメソッド呼び出しとして解釈されるので,演算子の追加や再定義も
可能である(\ref{section:operator}節).
演算子にあたるメソッドをローカル定義することで,グローバル環境を汚染しない
可読性の高い内部DSLを構築できる(\ref{section:dsl}節).
メソッド定義がクラス定義から独立していることは,対象のクラスに縛られないメソッドの
グループ化も可能にしている(\ref{section:grouping}節).
またパラメータ化されたモジュールであるトレイトを使えば,メソッドを共通化し
衝突を回避しながら再利用することができる(\ref{section:traits}節,
\ref{section:collision}節).


\chapter{関連研究}
\label{chapter:related-work}

Suzuのオブジェクトシステムおよびメソッドシステムがもたらすのと類似した柔軟性を持つ
従来の機構との関連性について述べる.

GluonJ\cite{GluonJ}はアスペクト指向プログラミングを支援するJavaの拡張である.
Glueと呼ばれるクラスを定義することで,既存のクラスに対しメソッドを追加・再定義する
などの拡張が可能である.
しかし拡張はグローバルに反映され,その影響範囲を限定することはできない.

C\#の拡張メソッド\cite{ExtentionMethods}は既存のクラスにメソッドを
追加したように見せることができる機能である.
実際に呼ばれるのは第1引数に\verb|this|と指定した静的メソッドである.
拡張メソッドは\verb|using|ディレクティブによって有効無効を制御できるが,
同じシグネチャを持つ拡張メソッドが複数存在する場合エラーが起き,
この衝突を個別に解決する手段を持たない.
また既に通常の方法でクラスに定義されているインスタンスメソッドを再定義することも
できない.

Scala\cite{Scala}のimplicit conversionも既存のクラスに対しメソッドを
追加したように見せられる機能である.
実際は存在しないメソッドが呼ばれた際に型変換を行うことで,別のクラスのメソッドを
呼び出している.
スコープはブロック単位で制御できるが,同じシグネチャのメソッドを提供する
型への変換が複数存在する場合エラーが起き,この衝突を個別に解決する手段を
持たない.
またこれも拡張メソッドと同様,既に通常の方法でクラスに定義されているメソッドを
再定義することはできない.

Haskellの型クラス\cite{TypeClasses}はオブジェクト指向プログラミングのための
機構ではないが,
データ型の定義と独立して型ごとに操作を定義できる点がSuzuのメソッドに類似している.
型クラスは導入に静的な型を必要とするが,Suzuのメソッドはこれを必要としない.
また型クラスにおいて操作の実体を定義するインスタンス宣言は
モジュールシステムによって個別にエクスポートするかどうかの指定ができない.
ただし型クラスは戻り値の型に応じて関数の振る舞いを変えさせることができるのに対し,
Suzuではこれは不可能である.

OCaml\cite{OCaml}のlocal openはSuzuと同様モジュール内の変数を局所的に
インポートできる.
これとユーザー定義演算子を組み合わせると演算子の意味を局所的に変えられる.
しかしOCamlにはオーバーロードが無いため演算子の振る舞いを型ごとに
変えることができず,既存の演算子を上書きしてしまうため内部DSLを構築するには
不便である.

Traits\cite{Traits,ApplyingTraits,FineGrainedReuse}はSuzuの
トレイトの元となった概念である.
メソッドの集合であるトレイトについて合成やリネームなどの演算を行うことで,
メソッドの衝突を回避しつつクラスにメソッドを提供できる.
Suzuはクラスではなくレキシカル環境にメソッドを定義するため,
トレイトを値を受け取ってモジュールを返す関数としてとらえ直した.
この方式では必要なメソッドと提供するメソッドが関数の引数と戻り値として明確となり,
モジュール演算によってメソッドの衝突を回避できるということを示した.

CLOS\cite{CLOS}はCommon Lispの言語仕様に含まれるオブジェクト指向
プログラミングの支援機構で,Suzuと同様メソッドをクラスとは独立して定義する.
メソッドはクラスではなく総称関数に対して追加され,総称関数は引数として
与えられたオブジェクトに応じて適切なメソッドを選択し呼び出す.
CLOSはCLtL2\cite{CLtL2}と呼ばれる仕様では既存の総称関数の内容を
コピーしてローカルな総称関数を定義する\verb|with-added-methods|が
含まれていたが,最新の仕様からは削除されている.
これはメソッドのローカル定義が有用でないと判断されたためである.
Suzuは非S式文法の言語として,ユーザー定義演算子と組み合わせることで
グローバル環境を汚染しない可読性の高い内部DSLを構築し,ローカルメソッドの
有効な利用法を示した.

Classbox\cite{Classboxes},Refinements\cite{Refinements},
Method Shells\cite{MethodShells}は,いずれも既存の
オブジェクト指向言語に対しモジュール単位でスコープを限定したメソッドの
追加・再定義を行うための機構である.
違いはメソッドのスコープルールであり,Classboxはダイナミックスコープ,
Refinementsはレキシカルスコープ,Method Shellsはそれらを切り替えられる.
SuzuのモジュールシステムはRefinementsに近い.ただしSuzuは
モジュール単位のみならずブロック単位でメソッドの追加・再定義が行え,
内部DSLの構築の際により利便性が高い.

MixJuice\cite{MixJuice}はクラス定義を複数のモジュールに分割して
書くことができる言語である.
MixJuiceによるモジュール分割はSuzuが行えるそれと非常に類似している.
MixJuiceにはパラメータ化されたモジュールがないがSuzuにはあり,
Suzuには静的な型検査がないがMixJuiceにはある.
またSuzuのメソッドのスコープはモジュール単位ではなくブロック単位であり
より細かい単位で制御できる.

\chapter{今後の課題}
\label{chapter:future-work}

Suzuの仕様と実装にまつわる今後の課題を述べる.

\section{継承機構}

Suzuにはクラスを継承する機能を実装していない.これはローカルメソッドと組み合わせた
際にどのように振舞うのが適切か定かではないためである.
クラス\verb|A|を継承したクラス\verb|B|があるとする.ネストしたブロックにおいて,
\begin{quote}
	\begin{verbatim}
	begin:
	  let B#m = ..
	  begin:
	    let A#m = ...
	    inst.m
	  end
	end
	\end{verbatim}
\end{quote}
のように,外側のブロックでサブクラス\verb|B|のメソッド\verb|m|を,
内側のブロックでスーパークラス\verb|A|のメソッド\verb|m|を定義し,
サブクラス\verb|B|のインスタンス\verb|inst|に対しメソッド\verb|m|を
呼び出した場合,どちらの\verb|m|が呼び出されるべきか.

つまり,内側に定義されたスーパークラスのメソッドと外側に定義されたサブクラスのメソッドの
どちらを優先すべきかという問題である.
この問題を解決するには,実際のプログラムにおいてこのようなケースが
生じた場合どちらが優先されてほしいかを調査する必要がある.


\section{多重ディスパッチ}

1つのオブジェクトのクラスに応じてメソッドの選択を行う単一ディスパッチに対し,
複数のオブジェクトのクラスに応じてメソッドの選択を行うのが多重ディスパッチである.
Suzuは単一ディスパッチにしか対応していないが,メソッドはクラスではなく
レキシカル環境に定義されているためディスパッチで考慮するクラスを
1つに限定する必要はなく,多重ディスパッチが可能なCLOSに倣えば原理上比較的容易に
多重ディスパッチに対応できると考えられる.

問題は適切なシンタックスを考案することである.内部DSLの構築にあたっては
語順も重要な要素なので,なるべく語順を崩すことのない文法が必要である.

\section{ダイナミックスコープ}

Suzuのメソッドは変数と同じレキシカルスコープである.
これにより演算子を用いた内部DSLを構築するにあたって,スコープ外の関数を呼び出した
際演算子の振る舞いが変わっているために結果がおかしくなってしまうようなことが起きない
ようになっている.

しかしながらダイナミックスコープであるほうが有用な場合もある.
例えば値を出力する関数\verb|print|が内部で値に対しメソッド\verb|to_string|
を呼び出しているとすると,ダイナミック環境の\verb|to_string|を再定義することで
局所的に出力内容を変えることができる.

レキシカルスコープ・ダイナミックスコープはともに有用であるため,ユースケースを分析し
適切な方法で共存させつつ導入する必要がある.

\section{オブジェクト固有のメソッド}

Rubyの特異メソッド,Common LispのEQLスペシャライザ,あるいはECMAScript
\cite{ECMAScript}のようなプロトタイプベースのオブジェクト指向言語を用いると,
オブジェクトに対しクラスによらない固有のメソッドを定義できる.
Suzuは現状そのような機能は持っておらず,オブジェクト固有のメソッドは定義できない.
しかしながらまずはこれら既存の機能の有用性自体を検証する必要がある.

\section{効率的な実装}

Suzuではブロックという細かい単位でメソッドの再定義ができる.
これはつまり同じクラスのオブジェクトに対し同じ名前のメソッドを呼び出しても,
プログラムのどの位置で呼び出したかによって呼び出されるメソッドが異なるということである,
これは従来のオブジェクト指向言語にはない性質であり,メソッド呼び出しに対する
既存の最適化手法は適用できない可能性がある.
現在Suzuの処理系はメソッド呼び出しに関して特に最適化を施しておらず,
適切な最適化手法を考え実装することが課題である.


\chapter{結論}
\label{chapter:conclusion}

ローカル変数に対応するローカルメソッドを定義可能なオブジェクト指向言語Suzuを開発し,
その有用性を実証した.

Suzuは変数名の代わりにクラス名とメソッド名の組を用いることで変数と同じ
シンタックスおよびセマンティクスでメソッドを定義でき,これによりメソッドを変数と同じように
ローカル定義することを可能にしている.
ローカルメソッドをユーザー定義演算子と組み合わせることで,グローバル環境を汚染せず
可読性の高い内部DSLを利用できることを示した.
また変数定義とメソッド定義のセマンティクスを統一することでモジュールシステムを統一でき,
メソッドの柔軟なグループ化・共通化・再利用が可能となることを示した.

従来の研究と比べてメソッドのスコープをブロックという細かい単位で制御でき,かつ
モジュールシステムによる衝突回避を利用できることがSuzuの優れている点である.
今後は他のオブジェクト指向言語が持つ機能への対応や効率的な実装が課題である.


\chapter*{謝辞}
\addcontentsline{toc}{chapter}{\numberline{}謝辞}

本研究を行うにあたり,多大なるご指導とご助言を下さった筑波大学システム情報系
前田敦司准教授に深く感謝いたします.
また第56回プログラミング・シンポジウムにて有益なコメントを下さった方々に
感謝いたします.
最後に貴重なご意見を下さった筑波大学インタラクティブ・アーキテクチャ研究室の皆様と
OBの水島宏太さんに感謝いたします.

\newpage

\nocite{Java}
\nocite{ContextJ}
\nocite{C3Linearization}
\nocite{CISCO}
\nocite{OptimizingMessageSends}
\nocite{MethodCachingForRuby}
\nocite{ShiftResetTutorial}
\nocite{ShiftResetOnMinCaml}
\nocite{ExnAndDelimCont}
\nocite{Containerless}
%\addcontentsline{toc}{chapter}{\numberline{}参考文献}
\renewcommand{\bibname}{参考文献}

%% 参考文献に jbibtex を使う場合
\bibliographystyle{junsrt}
\bibliography{thesis}
%% [compile] jbibtex sample; platex sample; platex sample;

%% 参考文献を直接ファイルに含めて書く場合
%\begin{thebibliography}{1}
%\bibitem{RakRak}
%野寺隆志.
%\newblock 楽々 \LaTeX.
%\newblock 共立出版, 1990.
%
%\bibitem{JiyuuJizai}
%磯崎秀樹.
%\newblock \LaTeX 自由自在.
%\newblock サイエンス社, July 1992.
%
%\bibitem{bryant-ieeetc86}
%Randal~E. Bryant.
%\newblock Graph-based algorithms for {B}oolean function manipulation.
%\newblock {\em IEEE Transactions on Computers}, Vol. C-35, No.~8, pp. 677--691,
%  August 1986.
%\end{thebibliography}

\end{document}
