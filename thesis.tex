% withpage: ページ番号をつける (著者確認用)
% english: 英語原稿用フォーマット
\documentclass{ipsjprosym}
%\documentclass[withpage,english]{ipsjprosym}

\usepackage[dvips]{graphicx}
\usepackage{latexsym}

\begin{document}

% Title, Author %%%%%%%%%%%%%%%%%%%%%%%%%%%%%%%%%
\title{環境にメソッドを直接格納する \\ 新しいオブジェクトシステムの提案}

\affiliate{COINS}{筑波大学情報科学類}
\affiliate{CS}{筑波大学システム情報系}

\author{林 拓人}{Takuto Hayashi}{COINS}[hayashi@ialab.cs.tsukuba.ac.jp]
\author{前田 敦司}{Atusi Maeda}{CS}[maeda@cs.tsukuba.ac.jp]

\begin{abstract}
[概要(400字程度)]
メソッドの格納場所という分類においてまったく新しいオブジェクトシステムを提案する.
これを実装したプログラミング言語Suzuを用い,他のオブジェクトシステムにおける機構との比較を交えて
本提案の意義を示す.
\end{abstract}

\begin{jkeyword}
オブジェクトシステム,メソッド,クラス,総称関数,環境
\end{jkeyword}

\maketitle

% Body %%%%%%%%%%%%%%%%%%%%%%%%%%%%%%%%%
\section{序論}

既存のオブジェクトシステムをメソッドの格納場所で分類

どの分類にも属さない新たなオブジェクトシステムを提案

これが有用であることを示す

\section{手法}

クラスでも総称関数でもなく環境にメソッドを直接格納する

変数に対して行える操作がすべてメソッドに対しても可能となる

多重ディスパッチや多重継承にも自然に拡張可能

\section{実装:プログラミング言語Suzu}

提案するオブジェクトシステムに加え,ブロック付き関数呼び出しやユーザ定義演算子,Traits,限定継続オペレータ等を実装

オブジェクトシステムの特徴を生かしたDSLの作成が可能

実例:PEGパーザコンビネータ

\section{比較}

クラスにメソッドを格納するオブジェクトシステムの拡張機構について,本提案がより自然な形で同等もしくはそれ以上の機能を提供していることを示す

Classboxes,Refinements,MethodShelters

拡張メソッド,型クラス,MixJuice

\section{メソッド呼び出しの最適化について(おまけ)}

\section{結論}

環境にメソッドを直接格納する新しいオブジェクトシステムを提案した

本提案はDSLの記述や既存のオブジェクトシステムの整理という面において有用である

% \begin{acknowledgment}
% 謝辞が必要であれば,ここに書く.
% \end{acknowledgment}

% BibTeX を使用する場合 %%%%%%%%%%%%%%%%%%%%%%%%%%%%%%%%%
% \bibliographystyle{ipsjsort}
% \bibliography{ref}

% BibTeX を使用しない場合
\begin{thebibliography}{9}
 \bibitem{latex} 奥村晴彦, 黒木裕介: \textbf{LaTeX2e美文書作成入門}. 技術評論社, 2013.
\end{thebibliography}

\end{document}
