% withpage: ページ番号をつける (著者確認用)
% english: 英語原稿用フォーマット
\documentclass{ipsjprosym}
%\documentclass[withpage,english]{ipsjprosym}

\usepackage[dvipdfmx]{graphicx}
\usepackage{latexsym}

\begin{document}

% Title, Author %%%%%%%%%%%%%%%%%%%%%%%%%%%%%%%%%
\title{環境にメソッドを直接格納する \\ 新しいオブジェクトシステムの提案}

\affiliate{COINS}{筑波大学情報学群情報科学類}
\affiliate{CS}{筑波大学システム情報系}

\author{林 拓人}{Takuto Hayashi}{COINS}[hayashi@ialab.cs.tsukuba.ac.jp]
\author{前田 敦司}{Atusi Maeda}{CS}[maeda@cs.tsukuba.ac.jp]

\begin{abstract}
[概要(400字程度)]
\end{abstract}

\begin{jkeyword}
オブジェクトシステム,メソッド,クラス,総称関数,環境
\end{jkeyword}

\maketitle

% Body %%%%%%%%%%%%%%%%%%%%%%%%%%%%%%%%%
\section{序論}

従来のクラスベース・オブジェクトシステムにおけるメソッドの格納方式は,モデル化すると大きく
2つに分類することができる.
1つはクラスにメソッドを格納する方式,もう1つはCLOS\cite{CLOS}のように総称関数にメソッドを
格納する方式である.

本論文は,これら2つの格納モデルに対する分析に基づき考案した,全く新しい格納モデルを持つ
オブジェクトシステムを提案する.
また,その特長を最大限に活かせるよう独自に設計したプログラミング言語Suzuを用いて,提案する
オブジェクトシステムの評価を行う.

\section{従来のオブジェクトシステム}

クラスにメソッドを格納するオブジェクトシステムは環境にクラス名をキーとしてクラスを格納し,
クラスにメソッド名をキーとしてメソッドを格納する.総称関数にメソッドを格納するオブジェクトシステムは
環境にメソッド名をキーとして総称関数を格納し,総称関数にクラス名をキーとしてメソッドを格納する
(単一ディスパッチの場合).どちらもクラス名とメソッド名が決まればメソッドが一意に定まる.

\section{提案するオブジェクトシステム}

提案するオブジェクトシステムではクラスや総称関数と言った枠を廃し,クラス名とメソッド名の組を
キーとして環境にメソッドを直接格納する.これにより,ブロック単位でのスコープの制御やシャドーイング,
モジュールからのエクスポート・インポート,仮引数やパターンマッチングのパターンとしての指定など,
変数に対して行えるあらゆる操作がメソッドに対しても行えるようになる.また拡張によって多重ディスパッチや
多重継承などに対応することもできる.

\section{実装}

\subsection{プログラミング言語Suzuの概要}

提案する手法の有用性を実証するため,独自に設計したプログラミング言語Suzuを実装した.
Suzuは,クラス・コンストラクタ・セッター・ゲッター・メソッドといったオブジェクト指向プログラミングにおける
種々の概念に加え,トレイトやユーザ定義演算子,ラベル付き引数,ブロック付き関数呼び出し,
限定継続などの機構を搭載している.これらの機構が,オブジェクトシステムの利点を生かし,
PEGパーザコンビネータのようなドメイン特化言語を作成する際に有用であることを示す.

\subsection{プログラム例}

\subsubsection{PEGパーザコンビネータ}

\section{比較}

クラスにメソッドを格納するオブジェクトシステムの拡張機構について,本提案がより自然な形で同等
もしくはそれ以上の機能を提供していることを示す.

クラスにメソッドを格納する方式のオブジェクトシステムにおいて提案するオブジェクトシステムに似た
柔軟性を実現するための既存の機構であるRefinementsやClassbox,MethodShelters,
拡張メソッド等との関連や,概念的に類似したHaskellの型クラスおよびMixJuiceとの関連についても
議論する.

\section{今後の課題}

\subsection{モデルの拡張}

\subsubsection{多重継承}

\subsubsection{多重ディスパッチ}

\subsection{メソッド呼び出しの最適化}

\section{結論}

環境にメソッドを直接格納する新しいオブジェクトシステムを提案した

本提案はDSLの記述や既存のオブジェクトシステムの整理という面において有用である

% \begin{acknowledgment}
% 謝辞が必要であれば,ここに書く.
% \end{acknowledgment}

% BibTeX を使用する場合 %%%%%%%%%%%%%%%%%%%%%%%%%%%%%%%%%
\bibliographystyle{ipsjsort}
\bibliography{thesis}

% BibTeX を使用しない場合
% \begin{thebibliography}{9}
% \bibitem{latex} 奥村晴彦, 黒木裕介: \textbf{LaTeX2e美文書作成入門}. 技術評論社, 2013.
% \end{thebibliography}

\end{document}
