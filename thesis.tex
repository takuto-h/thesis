%%
% このファイルは、筑波大学情報学群情報科学類の
% 卒業研究論文本体のサンプルです。
% このファイルを書き換えて、この例と同じような書式の論文本体を
% LaTeXを使って作成することができます。
% 
% PC環境や、LaTeX環境の設定によっては漢字コードや改行コードを
% 変更する必要があります。
%%
\documentclass[a4paper,11pt,dvipdfmx]{jreport}

%%【PostScript, JPEG, PNG等の画像の貼り込み】
%% 利用するパッケージを選んでコメントアウトしてください。
\usepackage{graphicx} % for \includegraphics[width=3cm]{sample.eps}
%\usepackage{epsfig} % for \psfig{file=sample.eps,width=3cm}
%\usepackage{epsf} % for \epsfile{file=sample.eps,scale=0.6}
%\usepackage{epsbox} % for \epsfile{file=sample.eps,scale=0.6}

%% dvipdfm を使う場合(dvi->pdfを直接生成する場合)
%\usepackage[dvipdfm]{color,graphicx}
%% dvipdfm を使ってPDFの「しおり」を付ける場合
%\usepackage[dvipdfm,bookmarks=true,bookmarksnumbered=true,bookmarkstype=toc]{hyperref}
%% 参考:dvipdfm 日本語版
%% http://hamilcar.phys.kyushu-u.ac.jp/~hirata/dvipdfm/

\usepackage[bookmarksnumbered=true]{hyperref}
\usepackage{pxjahyper}

\usepackage{times} % use Times Font instead of Computer Modern

\setcounter{tocdepth}{3}
\setcounter{page}{-1}

\setlength{\oddsidemargin}{0.1in}
\setlength{\evensidemargin}{0.1in} 
\setlength{\topmargin}{0in}
\setlength{\textwidth}{6in} 
%\setlength{\textheight}{10.1in}
\setlength{\parskip}{0em}
\setlength{\topsep}{0em}

%\newcommand{\zu}[1]{{\gt \bf 図\ref{#1}}}

%% タイトル生成用パッケージ(重要)
\usepackage{coins-jp}
\usepackage{jumoline}

%% タイトル
%% 【注意】タイトルの最後に\\ を入れるとエラーになります
\title{\Underline{レキシカル環境にメソッドを定義する\\オブジェクト指向言語Suzu}}
%% 著者
\author{林 拓人}
%% 指導教員
\advisor{前田敦司}

%% 専攻名 と 年月 (提出年月)
%% 年月は必要に応じて書き替えてください。
\heiseiyear{26}  % 平成の年度
\majorfield{ソフトウェアサイエンス主専攻}
%\majorfield{情報システム主専攻}
%\majorfield{知能情報メディア主専攻}

\makeatletter%% プリアンブルで定義する場合は必須

%% (j)report・(j)book クラスの場合
%% 
\renewenvironment{thebibliography}[1]% 再定義
{\chapter*{\bibname\@mkboth{\bibname}{\bibname}}%
	\addcontentsline{toc}{chapter}{\numberline{}\bibname}% この行追加
	\list{\@biblabel{\@arabic\c@enumiv}}%
	{\settowidth\labelwidth{\@biblabel{#1}}%
		\leftmargin\labelwidth
		\advance\leftmargin\labelsep
		\@openbib@code
		\usecounter{enumiv}%
		\let\p@enumiv\@empty
		\renewcommand\theenumiv{\@arabic\c@enumiv}}%
	\sloppy
	\clubpenalty4000
	\@clubpenalty\clubpenalty
	\widowpenalty4000%
	\sfcode`\.\@m}
{\def\@noitemerr
	{\@latex@warning{Empty `thebibliography' environment}}%
	\endlist}

\makeatother%% プリアンブルで定義する場合は必須


\begin{document}
\maketitle
\thispagestyle{empty}
\newpage

\thispagestyle{empty}
\vspace*{20pt plus 1fil}
\parindent=1zw
\noindent
%%
%% 論文の概要(Abstract)
%%
\begin{center}
{\Large \bf 要  旨}
\vspace{2cm}
\end{center}
[400字程度]

%%%%%
\par
\vspace{0pt plus 1fil}
\newpage

\pagenumbering{roman} % I, II, III, IV 
\tableofcontents
\listoffigures
%\listoftables

\pagebreak \setcounter{page}{1}
\pagenumbering{arabic} % 1,2,3


\chapter{序論}


\chapter{背景}


\chapter{実装:プログラミング言語Suzu}

\section{応用例}


\chapter{関連研究}


\chapter{結論}


\chapter*{謝辞}
\addcontentsline{toc}{chapter}{\numberline{}謝辞}

本研究を行うにあたり,多大なるご指導とご助言を下さった筑波大学システム情報系前田敦司准教授に
深く感謝いたします.
また第56回プログラミング・シンポジウムにて有益なコメントを下さった方々に感謝いたします.
最後に貴重なご意見を下さった筑波大学インタラクティブ・アーキテクチャ研究室の皆様とOBの水島宏太さんに
感謝いたします.

\newpage

%\addcontentsline{toc}{chapter}{\numberline{}参考文献}
\renewcommand{\bibname}{参考文献}

%% 参考文献に jbibtex を使う場合
\bibliographystyle{junsrt}
\bibliography{thesis}
%% [compile] jbibtex sample; platex sample; platex sample;

%% 参考文献を直接ファイルに含めて書く場合
%\begin{thebibliography}{1}
%\bibitem{RakRak}
%野寺隆志.
%\newblock 楽々 \LaTeX.
%\newblock 共立出版, 1990.
%
%\bibitem{JiyuuJizai}
%磯崎秀樹.
%\newblock \LaTeX 自由自在.
%\newblock サイエンス社, July 1992.
%
%\bibitem{bryant-ieeetc86}
%Randal~E. Bryant.
%\newblock Graph-based algorithms for {B}oolean function manipulation.
%\newblock {\em IEEE Transactions on Computers}, Vol. C-35, No.~8, pp. 677--691,
%  August 1986.
%\end{thebibliography}

\end{document}
